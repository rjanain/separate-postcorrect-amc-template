% =================================================================================
% AMC OMR Sheet Post-Correct Template: separate.tex
% Description: Generates a blank OMR answer sheet for Auto Multiple Choice.
% License: GPLv2+
% Author: Rakesh Jana
% Last Updated On: 4th May 2025
% =================================================================================

\documentclass[a4paper,11pt]{article}

% -------------------- Encoding and Language --------------------
\usepackage[utf8]{inputenc}
\usepackage[T1]{fontenc}
\usepackage[english]{babel}

% -------------------- AMC Package --------------------
\usepackage[separateanswersheet,insidebox,postcorrect,automarks]{automultiplechoice}

% -------------------- Layout --------------------
\usepackage{geometry}
\newgeometry{
	inner=20mm, outer=20mm,
	top=10mm, bottom=18mm,
	headheight=17mm,headsep=0mm,
	footskip=0mm,foot=12mm,
	marginparwidth=0mm,marginparsep=0mm,
	includehead,includefoot
}
\savegeometry{bubble}
\loadgeometry{bubble}

% -------------------- Extra Packages --------------------
\usepackage{amsmath,amssymb,amsfonts}
\usepackage{enumerate,multicol}
\usepackage{fp} % for Fixed Point Arithmetic
\usepackage{tikz}
\usepackage{totcount}

% -------------- Packages For Table --------------------
\usepackage{array}    % for >{\centering\arraybackslash}
\usepackage{makecell} % for \makecell

% Optional: define a C-column type for convenience
\newcolumntype{C}[1]{>{\centering\arraybackslash}p{#1}}

% -------------------- Question Counter --------------------
\usepackage{totcount}
\regtotcounter{AMCquestionaff}

% -------------------- AMC Layout--------------------
\setlength{\parindent}{0pt} 
%\setlength{\columnseprule}{.1pt}
\AMCcodeHspace=1em
\AMCcodeVspace=1em 
\AMCformHSpace=1em
\AMCformVSpace=2ex
\AMCboxStyle{shape=square,size=2.5ex,down=.4ex,rule=.5pt,outsidesep=.1em,color=black}

% -------------------- AMC Question Labels --------------------
\usepackage{xstring}
\def\mythresh{1} % Required for string comparison
\def\AMCchoiceLabelFormat#1{\color{black!50}\scriptsize{#1}}
\def\AMCbeginQuestion#1#2{\par\hspace{-1mm}{\sf Q-#1}$\rangle$#2 \hspace{2mm}}
\def\AMCformQuestion#1#2{\par\hspace{-1mm}\StrLen{#1}[\inputstrlen]
	{\sf Q-\ifthenelse{\inputstrlen>\mythresh}{}{0}#1}$\rangle$#2 \hspace{2mm}}

% ------------- AMC Command To be used in paper seting -------------
\makeatletter\newcommand{\slno}{\the\AMCid@etud}\makeatother
\makeatletter\newcommand{\makeAlph}[1]{\@Alph{#1}}\makeatother
\newcommand{\paperCode}{{\fontsize{50}{60}\selectfont \makeAlph{\slno}}}

% -------------------- User Parameters --------------------
\definecolor{mygray}{gray}{0.96}
\newcommand{\bbox}[1]{$\fcolorbox{black!40}{black!40}{\color{white}#1}$}
\def\iitem{$\bullet\kern4pt$}

% -------------------- Metadata --------------------
\newcommand{\examName}{A Sample Exam Name}
\newcommand{\departmentName}{Department of Mathemaics}
\newcommand{\instituteName}{Indian Institute of Technology Bombay}
\newcommand{\examDate}{May 4, 2023}
\newcommand{\examTime}{9:30 -- 11:30}


% -------------------- Exam Scoring --------------------
% Define scoring: S = single-correct; M = multiple-correct
% Parameter: mz = if all marked are correct, then mz, otherwise the score is zero.
% ==================== Scoring Parameters ==================
\def\scqmarks{6}	% Mark for each SCQ
\def\mcqmarks{6}	% Mark for each MCQ
\def\ntqmarks{5}	% Mark for each NTQ

% ==================== Scoring Defaults ==================
\def\scoringDefaultS{mz=\scqmarks}
\def\scoringDefaultM{mz=\mcqmarks}
\def\scoringDefaultN{\ntqmarks}  % for numeric answer no negative marking
\def\scoringDefaultNAPPROX{\ntqmarks}
\def\anynegativemarking{F} % N=No, H=Half, F=Full, Q=Quarter

% ------------ Formula Based Scoring --------------------
% =========================================================="
% The following variables are used in the scoring formula:	"
% N = number of proposed responses.							"
% NB = number of correct responses to the question.			"
% NBC = count of correct responses which have been checked.	"
% NM = number of wrong responses to the question.			"
% NMC = count of wrong responses which have been checked.	"
% =========================================================="
% NOTE: If you you are using formula based scoring, then you need to
% comment the above scoring parameters and uncomment the following
% scoring parameters as per your requirement.
% =========================================================="

% 1. Example: -3 if any wrong answer is marked, otherwise \scqmarks
% \def\scoringDefaultS{formula=(NMC > 0 ? -3 : \scqmarks}
% 2. Example: Award and deduct marks for each correct and wrong answer respectively.
%\def\scoringDefaultS{formula=(NBC/NB-NMC/NM)*\scqmarks}
% 3. Example: -3 if any wrong answer is marked, otherwise 
%   it awards marks based on number of correct answers marked.
%\def\scoringDefaultM{formula=(NMC>0 ? -3 : (NBC/NB)*\mcqmarks)}


% =================== Document =================== 
\begin{document}
	
% -------------------- Set Scoring Zone on OMR --------------------
\AMCsetScoreZoneAnswerSheet{width=1.5em, height=2ex, depth=.5ex, position=question}

% -------------------- Generate Question Paper --------------------
\onecopy{1}{  %%%%% Change 1 to number of paper variant needed.

	%%%%%%%%%% Header %%%%%%%%%%
	\begin{tabular*}{\textwidth}{@{\extracolsep{\fill}}
				C{0.9\textwidth}
				C{0.1\textwidth}
				}
		\makecell{
		{\examName}\\
		{\departmentName}\\
		{\instituteName}
		}%
		& \makecell{\fbox{\paperCode}} 
	\end{tabular*}
	%%%%%%%%%% END of HEADER %%%%%%%%%%

	%%%%%%%%%% Inser Dummy Questions %%%%%%%%%%
	% Replace `20` with total number of questions
	\foreach \qno in {1,2,...,20}{ 
	\begin{questionmult}{q\qno}
		\begin{choicescustom}
		\wrongchoice{}%
		\wrongchoice{}%
		\wrongchoice{}%
		\wrongchoice{}%
		%\wrongchoice{}%
		\end{choicescustom}
	\end{questionmult}
	}

	%%%%%%%%%% END OF Question Paper %%%%%%%%%%

\AMCcleardoublepage

% -------------------- Generate OMR Sheet --------------------
\AMCformBegin

%%%%%%%%%% OMR Header %%%%%%%%%%
\begin{tabular*}{\textwidth}{@{\extracolsep{\fill}}
				C{0.9\textwidth}
				C{0.1\textwidth}
			}
			\makecell{
			{\LARGE\scshape\bfseries Optical Mark Recognition (OMR) Sheet}\\
			{\LARGE \examName}\\
			{\LARGE \departmentName}\\
			{\LARGE\instituteName}
			}%
			& \makecell{\fbox{\paperCode}} 
\end{tabular*}

%%%%%%%%%% OMR Name Field %%%%%%%%%%
\begin{minipage}{\textwidth}
	\vspace*{7mm}
	\begin{minipage}{.7\linewidth}%namefield
	\namefield{ % starting point of namefield
		\begin{minipage}{0.95\linewidth}
		Name: \vspace*{1.75mm}\dotfill \\
		Reg. No.:  \bbox{\Large R} \bbox{\Large M} \bbox{\Large A} \bbox{\Large 2} \bbox{\Large 0}
		\bbox{\Large 2} \bbox{\Large 5}
		\boxed{\scalebox{2.5}{\phantom{0}}}\:
		\boxed{\scalebox{2.5}{\phantom{0}}}\:
		\boxed{\scalebox{2.5}{\phantom{0}}}\:
		\boxed{\scalebox{2.5}{\phantom{5}}}\:
		\boxed{\scalebox{2.5}{\phantom{9}}}
		\end{minipage}
	} % end of namefield
	\end{minipage}\hfill
	\begin{minipage}{.3\linewidth}
	Stud. Sign.: \vspace*{6mm}\dotfill  \\
	Invig. Sign.: \dotfill 
	\end{minipage}
\end{minipage}
%%%%%%%%%% End of OMR Name Field %%%%%%%%%%

\vspace*{0.5ex}\hrule\vspace*{1ex}

%%%%%%%%%% Registration Name Bubble Area and OMR Instruction %%%%%%%%%%

\begin{minipage}{\textwidth}
	\begin{minipage}[t]{.25\linewidth}
		% Registration Number Bubble Area
		\centering
		{\bf Bubble Last Five Digits of Reg. No.} \\ \hrule \vspace{2ex}
		\AMCcodeGridInt[top]{rollnumber}{5}
	\end{minipage}\hfill 
	\begin{minipage}[t]{.7\linewidth}
		%%%%% OMR Sheet Instruction %%%%%
		\textbf{\textsc{Instructions}}: \iitem Use only blue/black colored ball pen to make a bubble.
		
		\iitem The OMR sheet will be evaluated by using computer, so {\bf no writing is allowed anywhere in the OMR sheet except at dedicated places}.
		
		\iitem The right way to fill a box like \scalebox{1.25}{$\square$} is as  \scalebox{1.25}{$\blacksquare$}. Partially or improperly bubbled answers may not be considered during evaluation.
		
		\iitem Write your name and registration number on the OMR answer sheet in the space provided. Also put your signature on the OMR sheet in the dedicated space.
		
		\iitem Bubble your last five digits of your registration number, after ignoring the initial ``RMA2025'' part. For instance, if your registration number is RMA2025\underline{11223}, bubble the digits ``11223'' with the first digit in the first column and so on.
		%%%%% End of OMR Sheet Instruction %%%%%
	\end{minipage}
\end{minipage}

\vspace*{1ex}\hrule\hrule

%%%%% OMR Sheet Form %%%%%
		
\begin{multicols}{2}
\AMCform
\end{multicols}

} %%%%% End of Onecopy
\end{document}
